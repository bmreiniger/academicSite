\documentclass[addpoints,10pt]{exam}

\usepackage{amsmath,amsthm,amssymb}
\usepackage{fullpage}
\usepackage{enumerate}
\usepackage{tikz}\usetikzlibrary{calc}
%\usepackage{nth}
%\usepackage{graphicx}
%\usepackage{array}
%\usepackage{fancyvrb}%for code snippets but with math text

\newenvironment{graph}[1][scale=1]{
\begin{tikzpicture}[#1]
\tikzstyle{vertex}=[circle, draw, fill, inner sep=0pt, minimum size=4pt]%
\tikzstyle{bigvtx}=[circle, draw, fill, inner sep=0pt, minimum size=6pt]%
\tikzstyle{every path}=[line width=0.5pt]%
}{\end{tikzpicture}}

\newcommand{\cp}{\mathbin\Box}


%variables
\newcommand{\theclass}{Math 454 -- Graph Theory}
\newcommand{\themex}{2}
\newcommand{\thedate}{April 12, 2018}

\newcommand{\disp}{\displaystyle}
%\newcommand{\powerset}{\mathcal{P}}
\renewcommand{\emptyset}{\varnothing}
\newcommand{\ceil}[1]{\left\lceil#1\right\rceil}
\newcommand{\floor}[1]{\left\lfloor#1\right\rfloor}
\DeclareMathOperator{\rad}{rad}
\DeclareMathOperator{\diam}{diam}

%set up the header and footer
\pagestyle{headandfoot}
\header{Exam \themex}{\theclass}{\thedate}
\headrule
\setlength{\headsep}{0.25in}
\footer{}{Page \thepage}{}

% Create a True False question format
\newcommand*{\TrueFalse}[1]{%
\ifprintanswers
    \ifthenelse{\equal{#1}{T}}{%
        \textbf{TRUE}\hspace*{14pt}False
    }{
        True\hspace*{14pt}\textbf{FALSE}
    }
\else
    {True}\hspace*{20pt}False
\fi
} 
\newlength\TFlengthA
\newlength\TFlengthB
\settowidth\TFlengthA{\hspace*{1.36in}}
\newcommand\TFQuestion[2]{%
    \setlength\TFlengthB{\linewidth}
    \addtolength\TFlengthB{-\TFlengthA}
    \parbox[t]{\TFlengthA}{\TrueFalse{#1}}\parbox[t]{\TFlengthB}{#2}}
% Create a Yes No question format
\newcommand*{\YesNo}[1]{%
\ifprintanswers
    \ifthenelse{\equal{#1}{y}}{%
        \textbf{Yes}\hspace*{14pt}No
    }{
        Yes\hspace*{14pt}\textbf{NO}
    }
\else
    {Yes}\hspace*{20pt}No
\fi
} 
\newlength\YNlengthA
\newlength\YNlengthB
\settowidth\YNlengthA{\hspace*{1.16in}}
\newcommand\YNQuestion[2]{%
    \setlength\YNlengthB{\linewidth}
    \addtolength\YNlengthB{-\YNlengthA}
    \parbox[t]{\YNlengthA}{\YesNo{#1}}\parbox[t]{\YNlengthB}{#2}}


%before a paragraph, to indent all but the first line:
\newcommand{\hangpara}{
 \setlength{\parindent}{0cm} % don't indent new paragraphs
 \hangindent=0.7cm % indent all subsequent lines
}

\newcounter{savedqn}

%print the answers (or not)
%\printanswers

\begin{document}

\ifprintanswers
\begin{center}
	\textbf{Solutions}
\end{center}
\else
\vspace*{1em}
\makebox[0.9\textwidth]{Name: \hrulefill}

\vspace{20pt}

\begin{itemize}
\item \textbf{READ THE FOLLOWING DIRECTIONS!}
\item \textbf{Do NOT open the exam until instructed to do so.}
\end{itemize}
%\begin{minipage}[b][0.6\textheight][t]{0.65\textwidth}
\begin{itemize}
\item You have 
%choose one:
%until 12:45pm 
seventy-five (75) minutes
%
to complete this exam.  When you are told to stop writing, do it or you will lose all points on the page(s) you write on.
\item You may not communicate with other students during this test.
\item Keep your eyes on your own paper.
\item No written materials of any kind are allowed.  No scratch paper is allowed except as given by the proctor.
\item No phones, calculators, or any other electronic devices are allowed for any reason, including checking the time (a simple wristwatch is fine).
\item Any case of cheating will be taken extremely seriously.

\bigskip

\item Show all your work and explain your answers when appropriate.
\item Before turning in your exam, check to make certain you've answered all the questions.

\bigskip

%\item There are ??????seven (7) pages of questions.  That gives you ????????a little over ten (10) minutes per page.  Later questions are generally longer than earlier ones.

\end{itemize}
%\end{minipage}
%\hfill

\vfill

%\begin{minipage}[b][0.6\textheight][t]{0.3\textwidth}
\begin{center}
\gradetable[h][questions]
\end{center}
%\end{minipage}

%\vspace{40pt}
%Some possibly useful formulas:
%\begin{align*}
% \iint_R (\del_x n - \del_y m ) \, dx \, dy = \int_a^b (m x' + n y' )\, dt \\
% \cos^2 t = \frac{1}{2} (1+\cos(2t)) \\
% \sin^2 t = \frac{1}{2} (1-\cos(2t))
%\end{align*}


%\newpage
%\textbf{This page is left intentionally blank.}

%You may use it for scratch work, but \textbf{do not} remove it.
\newpage
\fi


\section*{Short answer}

\begin{questions}

\question[8] Find the maximum number of edges in a 16-vertex $K_4$-free graph.  Briefly justify.
\vfill

\question[12] Prove that the following graph $G$ cannot be properly colored from the displayed list assignment.  What does this say about $\chi_{\ell}(G)$?

\begin{graph}[scale=2]
\foreach \i in {0,1,2}
{
  \foreach \j in {0,1}
  {
    \node[vertex] (\i\j) at (\i,\j) {};
  }
}
\draw (11)--(01)--(00)--(10)--(11)--(21)--(20)--(10);
\node [label=above:{$\{1,3\}$}] at (01) {};
\node [label=above:{$\{1,2\}$}] at (11) {};
\node [label=above:{$\{2,4\}$}] at (21) {};
\node [label=below:{$\{2,3\}$}] at (00) {};
\node [label=below:{$\{1,2\}$}] at (10) {};
\node [label=below:{$\{1,4\}$}] at (20) {};
\end{graph}

\vfill
%\question The following bipartite graph \emph{does not have} a perfect matching.  Prove this (as efficiently as possible).

\question[12] Prove that the following graph is 4-critical.  \textit{(This is very quick with an appropriate theorem, but is not hard to do directly.)}

\begin{graph}[scale=1.25]
\node[vertex] (c) at (0,0) {};
\foreach \i in {1,2,...,5}
{
  \node[vertex] (w\i) at (90+72*\i:1) {};
  \draw (w\i)--(c);
}
\draw (w1)--(w2)--(w3)--(w4)--(w5)--(w1);
\end{graph}

\vfill

\newpage

\question[8] The matching displayed in the following graph is \emph{not} maximum.  Prove this by finding an augmenting path.

\begin{graph}
\foreach \i in {1,2,...,5}
{
  \node[vertex] (v\i) at (90-72*\i:1.5) {};
}
\node[vertex] (u1) at ($(v2)+(230:1.5)$) {};
\node[vertex] (u2) at ($(v2)+(270:1.5)$) {};
\node[vertex] (u3) at ($(v2)+(310:1.5)$) {};
  \node[vertex] (w1) at ($(v4)+(150:1.5)$) {};
  \node[vertex] (w2) at ($(v4)+(210:1.5)$) {};
\node[vertex] (c) at ($(v1)+(18:1.7)$) {};
\node[vertex] (a1) at ($(c)+(-18:1.7)$) {};
\node[vertex] (a2) at ($(a1)+(-60:1.7)$) {};
\node[vertex] (a3) at ($(a1)+(240:1.7)$) {};
  \node[vertex] (x) at ($(a1)+(18:1.7)$) {};
  \node[vertex] (y) at ($(a3)+(0,-1.5)$) {};
\draw (x)--(a1)--(a2)--(a3)--(a1)--(c)--(v1)--(v2)--(v3)--(v4)--(v5)--(v1);
\draw (y)--(a3);
\draw (u1)--(v2)--(u2); \draw (v2)--(u3);
\draw (w1)--(v4)--(w2);
\draw[line width=2pt,red] (v4)--(v5);
\draw[line width=2pt,red] (v2)--(v3);
\draw[line width=2pt,red] (v1)--(c);
\draw[line width=2pt,red] (a1)--(a3);
\end{graph}

\vfill


\question[15] Find a minimum vertex cover in the graph, and prove that it is minimum.

\begin{graph}
\foreach \i in {1,2,...,5}
{
  \node[vertex] (v\i) at (90-72*\i:1.5) {};
}
\node[vertex] (u1) at ($(v2)+(230:1.5)$) {};
\node[vertex] (u2) at ($(v2)+(270:1.5)$) {};
\node[vertex] (u3) at ($(v2)+(310:1.5)$) {};
  \node[vertex] (w1) at ($(v4)+(150:1.5)$) {};
  \node[vertex] (w2) at ($(v4)+(210:1.5)$) {};
\node[vertex] (c) at ($(v1)+(18:1.7)$) {};
\node[vertex] (a1) at ($(c)+(-18:1.7)$) {};
\node[vertex] (a2) at ($(a1)+(-60:1.7)$) {};
\node[vertex] (a3) at ($(a1)+(240:1.7)$) {};
  \node[vertex] (x) at ($(a1)+(18:1.7)$) {};
  \node[vertex] (y) at ($(a3)+(0,-1.5)$) {};
\draw (x)--(a1)--(a2)--(a3)--(a1)--(c)--(v1)--(v2)--(v3)--(v4)--(v5)--(v1);
\draw (y)--(a3);
\draw (u1)--(v2)--(u2); \draw (v2)--(u3);
\draw (w1)--(v4)--(w2);
\end{graph}

\vfill

\setcounter{savedqn}{\thequestion}
\end{questions}

%\section*{Algorithms}
%Give brief justifications for your answers (but not necessarily full proofs; a computation with each step clearly written may suffice).

%\begin{questions}
%\setcounter{question}{\thesavedqn}

%\question erm, do we have any algorithms this time?  Greedy Coloring...


%\setcounter{savedqn}{\thequestion}
%\end{questions}

\pagebreak

\section*{Proofs}
\begin{questions}
\setcounter{question}{\thesavedqn}
%\part A graph $G$ is bipartite if and only if $G\cp K_2$ is bipartite.
%\part Find (with proof) the largest number of edges in a $C_4$-free subgraph of $Q_3$.
%HOW TO PHRASE?  
%Let $G$ be a connected weighted graph with an edge $e$ of weight zero (and every other edge having positive weight).  How are the problems MinSpTree, MinDist(u,v), ChinesePostman on $G$ related to those on $G-e$, $G\cdot e$, $G+(\text{copy of $e$ with weight 0})$, $G+(\text{copy of $e$ with positive weight})$?

\question Let $G$ be an $X,Y$-bigraph with $n$ vertices.  Please note that throughout this problem we are using $\alpha(G)$, not $\alpha'(G)$.
\begin{parts}
%3.1.40, but the solution there uses alpha+beta=n
\part[6] Prove that $\alpha(G)\geq n/2$.
\vfill
\part[10] Prove that if $G$ has no perfect matching, then $\alpha(G)>n/2$.  \textit{(Suggestion: first, is $|X|=|Y|$?  Then use Hall's Theorem.)}
%If G is not balanced, then alpha>n/2; so suppose G is balanced.
%If G has no perfect matching, then there is an S\subseteq X$ with $|N(S)|<|S|$.  Now S\cup (Y-N(S)) is an independent set, and has size |S|+(n/2)-|N(S)|>n/2.
\vfill\vfill
\part[10] Prove that if $\alpha(G)>n/2$, then $G$ has no perfect matching.  \textit{(Suggestion: use Tutte's 1-Factor Theorem.)}
%Let I be an independent set of size >n/2, and consider S=V(G)-I.  Then G-S has |I| odd components (isolated vertices) and |S|<n/2<|I|=o(G-S).
\vfill\vfill
\end{parts}
%always for bipartite, \alpha \geq n/2.   if \alpha>n/2, then either not balanced hence no p.m., or there such set I intersects both X and Y.  Say |I\cap X|\geq |I\cap Y|.  Let S=I\cap X; then N(S)\subseteq Y\setminus I, so |N(S)|\leq |Y-I| = |Y|-|Y\cap I| \leq |Y|-|S|......

\newpage

\question[24] Prove \textbf{three} of the following four statements.  \textit{(All of these have short proofs, some \emph{very} short.)}
\begin{enumerate}[(i)]
\item %5.2.1
If $G$ is a graph such that $\chi(G-x-y)=\chi(G)-2$ for every pair of distinct vertices $x,y$, then $G$ is a complete graph.
%suppose to the contrary that G is not complete: there exists xy\notin E(G).  And suppose that G-x-y is (\chi(G)-2)-colorable.  Then we may use one additional color on each of x and y, for a (\chi(G)-1)-coloring of G, a contradiction.

%\question Suppose a graph has degree sequence 6,6,6,4,3,3,3,3.  Can you determine its chromatic number?  (If not, give the best bounds you can.)   
%\leq 3 by greedy; cannot be bipartite because only one degree not divisible by 3
%is graphic?  5533222, 422211, 11110, yep

\item Always $\chi(G)\cdot \chi(\overline{G})\geq n(G)$.
%\question Prove that $\chi(G\cup H) \leq \chi(G)\chi(H)$.
%pf1: \chi(G)\geq n(G)/alpha(G) = n(G)/\omega(\overline{G}) \geq n(G)/\chi(\overline{G}).
%pf2: consider optimal colorings of f of G and g of \overline{G}, and at each vertex v consider the ordered pair h(v):=(f(v), g(v)).  For each edge xy of G, f(x)\neq f(y), so h(x)\neq h(y); and for each nonedge xy of G, xy is an edge in \overline{G}, so g(x)\neq g(y), so h(x)\neq h(y).  Thus, h is an injection from V into [\chi(G)]\times[\chi(H)].

%\question %3.1.19
%....system of distinct representatives....

\item %(3.3.10) 
Always $\beta(G)\leq 2\alpha'(G)$. % For each $k\in\mathbb{N}$, construct a simple graph $G$ with $\alpha'(G)=k$ and $\beta(G)=2k$.
%Consider a maximum matching M.  Then the set of endpoints of edges in M has size 2\alpha', and is a vertex cover: if any edge is not covered, then it could be added to M, contradicting its maximality.

\item %(3.1.9)
Every maximal matching in a graph $G$ has size at least $\alpha'(G)/2$.
%pf1: Let M be maximal and M^* maximum matching.  Consider M\symmdiff M^*: each component is an even cycle or a path.  An even cycle has an equal number of M- and M^*-edges, and any path of length at least two has at most twice as many M^*-edges as M-edges.  Only a component consisting of one M^*-edge contributes more, but in this case such an edge could be added to M, contradicting its maximality.
%pf2: Let M be maximal and M^* maximum matching.  Suppose to the contrary that |M^*|>2|M|.  Then M saturates fewer than $M^*$ vertices, and hence some edge of M^* can be added to M, contradicting maximality of M.
%pf3: (similar to pf2). Let M be maximal matching; the endpoints of M form a vertex cover (if an edge is uncovered, then it could be added to M).  So 2|M|\geq \beta \geq \alpha'.

\end{enumerate}



%(The next are probably better as T/F, with just short justification)
%\question %5.1.12
%Prove or disprove: every optimal coloring of $G$ has a color class of size $\alpha(G)$.
%Prove or disprove: every $k$-chromatic graph $G$ has a proper $k$-coloring with a color class of size $\alpha(G)$.
%\question %5.1.15
%Prove or disprove: if $a$ is the average degree of $G$, then $\chi(G)\leq 1+a$.

\end{questions}

\newpage

(The statements are repeated here for your convenience.)
\begin{enumerate}[(i)]
\item %5.2.1
If $G$ is a graph such that $\chi(G-x-y)=\chi(G)-2$ for every pair of distinct vertices $x,y$, then $G$ is a complete graph.
\item Always $\chi(G)\cdot \chi(\overline{G})\geq n(G)$.
\item %(3.3.10) 
Always $\beta(G)\leq 2\alpha'(G)$. % For each $k\in\mathbb{N}$, construct a simple graph $G$ with $\alpha'(G)=k$ and $\beta(G)=2k$.
\item %(3.1.9)
Every maximal matching in a graph $G$ has size at least $\alpha'(G)/2$.
\end{enumerate}






%\ifprintanswers{}\else{
%\newpage
%\textbf{Scratch Paper - Do Not Remove}
\newpage
\textbf{Scratch Paper} - you may remove this if you find it convenient
\newpage
\textbf{Scratch Paper} - you may remove this if you find it convenient
%\newpage
%\textbf{Scratch Paper} - you may remove this if you find it convenient
%\newpage
%\textbf{Scratch Paper} - you may remove this if you find it convenient
%\newpage\hspace{-1em}\includegraphics{Ch1Tables.pdf}
%} \fi





\end{document}

