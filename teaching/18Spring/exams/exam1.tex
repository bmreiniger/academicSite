\documentclass[addpoints,10pt]{exam}

\usepackage{amsmath,amsthm,amssymb}
\usepackage{fullpage}
\usepackage{enumerate}
\usepackage{tikz}\usetikzlibrary{calc}
%\usepackage{nth}
%\usepackage{graphicx}
%\usepackage{array}
%\usepackage{fancyvrb}%for code snippets but with math text

\newenvironment{graph}[1][scale=1]{
\begin{tikzpicture}[#1]
\tikzstyle{vertex}=[circle, draw, fill, inner sep=0pt, minimum size=4pt]%
\tikzstyle{bigvtx}=[circle, draw, fill, inner sep=0pt, minimum size=6pt]%
\tikzstyle{every path}=[line width=0.5pt]%
}{\end{tikzpicture}}

\newcommand{\cp}{\mathbin\Box}


%variables
\newcommand{\theclass}{Math 454 -- Graph Theory}
\newcommand{\themex}{1}
\newcommand{\thedate}{March 8, 2018}

\newcommand{\disp}{\displaystyle}
%\newcommand{\powerset}{\mathcal{P}}
\renewcommand{\emptyset}{\varnothing}
\newcommand{\ceil}[1]{\left\lceil#1\right\rceil}
\newcommand{\floor}[1]{\left\lfloor#1\right\rfloor}
\DeclareMathOperator{\rad}{rad}
\DeclareMathOperator{\diam}{diam}

%set up the header and footer
\pagestyle{headandfoot}
\header{Exam \themex}{\theclass}{\thedate}
\headrule
\setlength{\headsep}{0.25in}
\footer{}{Page \thepage}{}

% Create a True False question format
\newcommand*{\TrueFalse}[1]{%
\ifprintanswers
    \ifthenelse{\equal{#1}{T}}{%
        \textbf{TRUE}\hspace*{14pt}False
    }{
        True\hspace*{14pt}\textbf{FALSE}
    }
\else
    {True}\hspace*{20pt}False
\fi
} 
\newlength\TFlengthA
\newlength\TFlengthB
\settowidth\TFlengthA{\hspace*{1.36in}}
\newcommand\TFQuestion[2]{%
    \setlength\TFlengthB{\linewidth}
    \addtolength\TFlengthB{-\TFlengthA}
    \parbox[t]{\TFlengthA}{\TrueFalse{#1}}\parbox[t]{\TFlengthB}{#2}}
% Create a Yes No question format
\newcommand*{\YesNo}[1]{%
\ifprintanswers
    \ifthenelse{\equal{#1}{y}}{%
        \textbf{Yes}\hspace*{14pt}No
    }{
        Yes\hspace*{14pt}\textbf{NO}
    }
\else
    {Yes}\hspace*{20pt}No
\fi
} 
\newlength\YNlengthA
\newlength\YNlengthB
\settowidth\YNlengthA{\hspace*{1.16in}}
\newcommand\YNQuestion[2]{%
    \setlength\YNlengthB{\linewidth}
    \addtolength\YNlengthB{-\YNlengthA}
    \parbox[t]{\YNlengthA}{\YesNo{#1}}\parbox[t]{\YNlengthB}{#2}}


%before a paragraph, to indent all but the first line:
\newcommand{\hangpara}{
 \setlength{\parindent}{0cm} % don't indent new paragraphs
 \hangindent=0.7cm % indent all subsequent lines
}

\newcounter{savedqn}

%print the answers (or not)
%\printanswers

\begin{document}

\ifprintanswers
\begin{center}
	\textbf{Solutions}
\end{center}
\else
\vspace*{1em}
\makebox[0.9\textwidth]{Name: \hrulefill}

\vspace{20pt}

\begin{itemize}
\item \textbf{READ THE FOLLOWING DIRECTIONS!}
\item \textbf{Do NOT open the exam until instructed to do so.}
\end{itemize}
%\begin{minipage}[b][0.6\textheight][t]{0.65\textwidth}
\begin{itemize}
\item You have 
%choose one:
%until 12:45pm 
seventy-five (75) minutes
%
to complete this exam.  When you are told to stop writing, do it or you will lose all points on the page(s) you write on.
\item You may not communicate with other students during this test.
\item Keep your eyes on your own paper.
\item No written materials of any kind are allowed.  No scratch paper is allowed except as given by the proctor.
\item No phones, calculators, or any other electronic devices are allowed for any reason, including checking the time (a simple wristwatch is fine).
\item Any case of cheating will be taken extremely seriously.

\bigskip

\item Show all your work and explain your answers when appropriate.
\item Before turning in your exam, check to make certain you've answered all the questions.

\bigskip

%\item There are ??????seven (7) pages of questions.  That gives you ????????a little over ten (10) minutes per page.  Later questions are generally longer than earlier ones.

\end{itemize}
%\end{minipage}
%\hfill

\vfill

%\begin{minipage}[b][0.6\textheight][t]{0.3\textwidth}
\begin{center}
\gradetable[h][questions]
\end{center}
%\end{minipage}

%\vspace{40pt}
%Some possibly useful formulas:
%\begin{align*}
% \iint_R (\del_x n - \del_y m ) \, dx \, dy = \int_a^b (m x' + n y' )\, dt \\
% \cos^2 t = \frac{1}{2} (1+\cos(2t)) \\
% \sin^2 t = \frac{1}{2} (1-\cos(2t))
%\end{align*}


%\newpage
%\textbf{This page is left intentionally blank.}

%You may use it for scratch work, but \textbf{do not} remove it.
\newpage
\fi


\section*{Short answer}

\begin{questions}

\question[12] Determine whether the following two graphs are isomorphic.  (Give an isomorphism or a short argument why they are not isomorphic.)

\begin{graph}
\foreach \i in {0,1,...,9}
{
  \node[vertex] (v\i) at (36*\i:2) {};
  \node at (36*\i:2.3) {\i};
  \draw (v\i) to (180+36*\i:2);
  \draw (v\i) to (36+36*\i:2);
}
\end{graph}
\qquad
\begin{graph}
\foreach \i/\j in {0/a,1/b,2/c,3/d,4/e,5/f}
{
  \node[vertex] (\j) at (60*\i:2) {};
  \node at (60*\i:2.3) {$\j$};
  \draw (\j) to (60+60*\i:2);
}
\foreach \i/\j in {0/g,1/h,2/i,3/j}
{
  \node[vertex] (\j) at (45+90*\i:1) {};
  \node at (45+90*\i:1.3) {$\j$};
}
\draw (a)--(j)--(e);
\draw (f)--(i)--(d);
\draw (i)--(g)--(h)--(j);
\draw (c)--(h); \draw (b)--(g);
\end{graph}

\vfill

\question[12] Decompose $K_5$ into one copy of each of the four trees below.

\begin{graph}
\node[vertex] (u) at (0,0) {}; \node[vertex] (v) at (1,0) {}; \draw (u)--(v);
\end{graph}
\qquad
\begin{graph}
\node[vertex] (v1) at (0,0) {};
\node[vertex] (v2) at (0.5,1) {};
\node[vertex] (v3) at (1,0) {};
\draw (v1)--(v2)--(v3);
\end{graph}
\qquad
\begin{graph}
\node[vertex] (v1) at (0,0) {};
\node[vertex] (v2) at (0.5,1) {};
\node[vertex] (v3) at (1,0) {};
\node[vertex] (v4) at (1.5,1) {};
\draw (v1)--(v2)--(v3)--(v4);
\end{graph}
\qquad
\begin{graph}
\node[vertex] (v1) at (0,0) {};
\node[vertex] (v2) at (0.5,0) {};
\node[vertex] (v3) at (1,0) {};
\node[vertex] (w) at (0.5,1) {};
\node[vertex] (l) at (1.5,1) {};
\draw (v1)--(w)--(v2);
\draw (w)--(v3)--(l);
\end{graph}

%\question Find a graceful labeling of the following graph.  (Reminder: vertex labels $\{0,\dotsc,8\}$ that induce edge ``lengths'' $\{1,\dotsc,8\}$.)

%\begin{graph}
%\node[vertex] (u) at (0,0) {};
%\foreach \i in {0,1,2}
%{
%  \node[vertex] (u\i) at (-1,1-\i) {};
%  \draw (u\i) to (u);
%}
%\node[vertex] (v) at (1,0) {};
%\foreach \i in {0,1,2,3}
%{
%  \node[vertex] (v\i) at (2,1-2/3*\i) {};
%  \draw (v\i) to (v);
%}
%\draw (u)--(v);
%\end{graph}

\vfill

\newpage

\setcounter{savedqn}{\thequestion}
\end{questions}

\section*{Algorithms}
Give brief justifications for your answers (but not necessarily full proofs; a computation with each step clearly written may suffice).

\begin{questions}
\setcounter{question}{\thesavedqn}

\question[12] Determine whether the following sequences are graphic (the degree sequence of a simple graph).
\begin{parts}
\begin{minipage}{0.5\textwidth}
\part 6 5 5 5 2 2 2 1 1
% odd sum
\end{minipage}
\begin{minipage}{0.5\textwidth}
\part 6 5 5 5 2 2 2 1
% 4 4 4 1 1 1 1
% 3 3 1 1 0 0
% 2 0 0 0 0
\end{minipage}
\vfill
%\part 6 5 5 4 3 2 2 1
% 4 4 3 2 1 1 1
% 3 2 1 1 1 0
% 1 1 0 0 0
\end{parts}

%\question
%\begin{parts}
%\part Find the Pr\"ufer code for the following tree.
%
%\begin{graph}[scale=1.25]
%\node[vertex] (1) at (0,0) {}; \node at (0,-.3) {1};
%\node[vertex] (2) at (0,1) {}; \node at (0,1.3) {2};
%\node[vertex] (3) at (0.5,0.5) {};\node at (0.5,0.2) {3};
%\node[vertex] (4) at (1,1) {}; \node at (1,1.3) {4};
%\node[vertex] (5) at (1,0) {}; \node at (1,-.3) {5};
%\node[vertex] (6) at (2,0) {}; \node at (2,-.3) {6};
%\draw (1)--(3)--(2); \draw (4)--(3)--(5)--(6);
%
%\end{graph}
%\part Find the tree with vertex set $[7]$ and Pr\"ufer code 62142.
%\end{parts}

\question[12] How many spanning trees does the following graph have?

\begin{graph}[scale=1.25]
\foreach \i in {0,1,2,3}
{
  \node[vertex] (v\i) at (45-90*\i:1) {};
}
\draw (v0)--(v1)--(v2)--(v3)--(v0);
\draw (v0)--(v2);
\draw[bend left] (v0) to (v1);
\draw[bend left] (v1) to (v0);
\draw[bend left] (v3) to (v0);
\end{graph}

\vfill 
\newpage

\question[12] \textit{Weighted graph algorithms}
\begin{parts}
\part Use either Kruskal's or Prim's algorithm to find a minimum spanning tree in the following weighted graph.

\begin{graph}
\node[vertex] (w) at (0,0) {};
%i indexes the vertices, j is the "next" cycle edge, l is the spoke edge
\foreach \i/\j/\l in {0/3/5,1/2/1,2/4/4,3/3/6,4/5/7,5/7/13}
{
  \node[vertex] (v\i) at (60*\i:2) {};
  \draw (v\i)--(60*\i+60:2);
  \node at (60*\i+30:1.9) {\j};
  \draw (w)--(v\i);
  \node at (60*\i+10:1) {\l};
}
\end{graph}
\vfill
\part Use Dijkstra's algorithm to find the minimum distances from the central vertex to each other vertex.
\begin{graph}
\node[vertex] (w) at (0,0) {};
%i indexes the vertices, j is the "next" cycle edge, l is the spoke edge
\foreach \i/\j/\l in {0/3/5,1/2/1,2/4/4,3/3/6,4/5/7,5/7/13}
{
  \node[vertex] (v\i) at (60*\i:2) {};
  \draw (v\i)--(60*\i+60:2);
  \node at (60*\i+30:1.9) {\j};
  \draw (w)--(v\i);
  \node at (60*\i+10:1) {\l};
}
\end{graph}
\vfill

\end{parts}

\newpage

\setcounter{savedqn}{\thequestion}
\end{questions}

\section*{Proofs}
%\part A graph $G$ is bipartite if and only if $G\cp K_2$ is bipartite.
%\part Find (with proof) the largest number of edges in a $C_4$-free subgraph of $Q_3$.
%HOW TO PHRASE?  
%Let $G$ be a connected weighted graph with an edge $e$ of weight zero (and every other edge having positive weight).  How are the problems MinSpTree, MinDist(u,v), ChinesePostman on $G$ related to those on $G-e$, $G\cdot e$, $G+(\text{copy of $e$ with weight 0})$, $G+(\text{copy of $e$ with positive weight})$?

\begin{questions}
\setcounter{question}{\thesavedqn}
\question[20] \textit{Digraphs}
\begin{parts}
\part Prove that if $D$ is a digraph with $\delta^+(D)\geq1$, then $D$ has a (directed) cycle.
\vfill\vfill
\part Use the statement in part~(a) to prove that if $D$ is a digraph with $d^+(v)=d^-(v)$ at every vertex $v$, then $D$ decomposes into (directed) cycles.
\vfill\vfill\vfill
\end{parts}

\newpage

\question[40]
Give complete careful proofs of \textbf{2 of the following 3} statements.  
\begin{enumerate}[(i)]
\item If $G$ is disconnected, then $\overline{G}$ is connected.
\item 
%2.1.3: 
A graph is a tree if and only if it is loopless and has exactly one spanning tree.
\item 
%1.3.12, first part:  
If $G$ is even (i.e., every vertex degree is even), then $G$ has no cut edge.
%Three good proofs: (1) get Eulerian circuit (of every component), removing an edge yields Eulerian paths->connected; (2) decompose into cycles, again removing an edge preserves connectedness; (3) by degrees, if cut edge e, the components of G-e have exactly one odd vertex.
\end{enumerate}

\newpage

(The statements are repeated here for your convenience.)
\begin{enumerate}[(i)]
\item If $G$ is disconnected, then $\overline{G}$ is connected.
\item A graph is a tree if and only if it is loopless and has exactly one spanning tree.
\item If $G$ is even (i.e., every vertex degree is even), then $G$ has no cut edge.
\end{enumerate}





%\ifprintanswers{}\else{
\newpage
\textbf{Scratch Paper - Do Not Remove}
%\newpage
%\textbf{Scratch Paper} - you may remove this if you find it convenient
%\newpage
%\textbf{Scratch Paper} - you may remove this if you find it convenient
%\newpage
%\textbf{Scratch Paper} - you may remove this if you find it convenient
%\newpage
%\textbf{Scratch Paper} - you may remove this if you find it convenient
%\newpage\hspace{-1em}\includegraphics{Ch1Tables.pdf}
%} \fi



\end{questions}


\end{document}

