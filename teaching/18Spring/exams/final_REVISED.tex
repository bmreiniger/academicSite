\documentclass[addpoints,10pt]{exam}

\usepackage{amsmath,amsthm,amssymb}
\usepackage{fullpage}
\usepackage{enumerate}
\usepackage{tikz}\usetikzlibrary{calc,decorations.markings,arrows.meta}
%\usepackage{nth}
%\usepackage{graphicx}
%\usepackage{array}
%\usepackage{fancyvrb}%for code snippets but with math text

\newenvironment{graph}[1][scale=1]{
\begin{tikzpicture}[#1]
\tikzstyle{vertex}=[circle, draw, fill, inner sep=0pt, minimum size=4pt]%
\tikzstyle{bigvtx}=[circle, draw, fill, inner sep=0pt, minimum size=6pt]%
\tikzstyle{every path}=[line width=0.5pt]%
}{\end{tikzpicture}}

\newcommand{\cp}{\mathbin\Box}


%variables
\newcommand{\theclass}{Math 454 -- Graph Theory}
%\newcommand{\themex}{2}
\newcommand{\thedate}{May 3, 2018}

\newcommand{\disp}{\displaystyle}
%\newcommand{\powerset}{\mathcal{P}}
\renewcommand{\emptyset}{\varnothing}
\newcommand{\ceil}[1]{\left\lceil#1\right\rceil}
\newcommand{\floor}[1]{\left\lfloor#1\right\rfloor}
\DeclareMathOperator{\rad}{rad}
\DeclareMathOperator{\diam}{diam}

%set up the header and footer
\pagestyle{headandfoot}
\header{Final Exam}{\theclass}{\thedate}
\headrule
\setlength{\headsep}{0.25in}
\footer{}{Page \thepage}{}

% Create a True False question format
\newcommand*{\TrueFalse}[1]{%
\ifprintanswers
    \ifthenelse{\equal{#1}{T}}{%
        \textbf{TRUE}\hspace*{14pt}False
    }{
        True\hspace*{14pt}\textbf{FALSE}
    }
\else
    {True}\hspace*{20pt}False
\fi
} 
\newlength\TFlengthA
\newlength\TFlengthB
\settowidth\TFlengthA{\hspace*{1.36in}}
\newcommand\TFQuestion[2]{%
    \setlength\TFlengthB{\linewidth}
    \addtolength\TFlengthB{-\TFlengthA}
    \parbox[t]{\TFlengthA}{\TrueFalse{#1}}\parbox[t]{\TFlengthB}{#2}}
% Create a Yes No question format
\newcommand*{\YesNo}[1]{%
\ifprintanswers
    \ifthenelse{\equal{#1}{y}}{%
        \textbf{Yes}\hspace*{14pt}No
    }{
        Yes\hspace*{14pt}\textbf{NO}
    }
\else
    {Yes}\hspace*{20pt}No
\fi
} 
\newlength\YNlengthA
\newlength\YNlengthB
\settowidth\YNlengthA{\hspace*{1.16in}}
\newcommand\YNQuestion[2]{%
    \setlength\YNlengthB{\linewidth}
    \addtolength\YNlengthB{-\YNlengthA}
    \parbox[t]{\YNlengthA}{\YesNo{#1}}\parbox[t]{\YNlengthB}{#2}}


%before a paragraph, to indent all but the first line:
\newcommand{\hangpara}{
 \setlength{\parindent}{0cm} % don't indent new paragraphs
 \hangindent=0.7cm % indent all subsequent lines
}

\newcounter{savedqn}

%print the answers (or not)
%\printanswers

\begin{document}

\ifprintanswers
\begin{center}
	\textbf{Solutions}
\end{center}
\else
\vspace*{1em}
\makebox[0.9\textwidth]{Name: \hrulefill}

\vspace{20pt}

\begin{itemize}
\item \textbf{READ THE FOLLOWING DIRECTIONS!}
\item \textbf{Do NOT open the exam until instructed to do so.}
\end{itemize}
%\begin{minipage}[b][0.6\textheight][t]{0.65\textwidth}
\begin{itemize}
\item You have 
%choose one:
%until 12:45pm 
two hours
%
to complete this exam.  When you are told to stop writing, do it or you will lose all points on the page(s) you write on.
\item You may not communicate with other students during this test.
\item Keep your eyes on your own paper.
\item No written materials of any kind are allowed.  No scratch paper is allowed except as given by the proctor.
\item No phones, calculators, or any other electronic devices are allowed for any reason, including checking the time (a simple wristwatch is fine).
\item Any case of cheating will be taken extremely seriously.

\bigskip

\item Show all your work and explain your answers when appropriate.
\item Before turning in your exam, check to make certain you've answered all the questions.

\bigskip

%\item There are ??????seven (7) pages of questions.  That gives you ????????a little over ten (10) minutes per page.  Later questions are generally longer than earlier ones.

\end{itemize}
%\end{minipage}
%\hfill

\vfill

%\begin{minipage}[b][0.6\textheight][t]{0.3\textwidth}
\begin{center}
\gradetable[h][questions]
\end{center}
%\end{minipage}

%\vspace{40pt}
%Some possibly useful formulas:
%\begin{align*}
% \iint_R (\del_x n - \del_y m ) \, dx \, dy = \int_a^b (m x' + n y' )\, dt \\
% \cos^2 t = \frac{1}{2} (1+\cos(2t)) \\
% \sin^2 t = \frac{1}{2} (1-\cos(2t))
%\end{align*}


%\newpage
%\textbf{This page is left intentionally blank.}

%You may use it for scratch work, but \textbf{do not} remove it.
\newpage
\fi


\section*{Short answer}
\begin{questions}


\question[20] Find all maximal paths, independent sets, cliques, paths, and matchings in the below graph.
\begin{minipage}{0.6\textwidth}
\vspace{1em}
\begin{enumerate}[(i)]
\item Maximal independent sets:
\vspace{3em}
\item Maximal cliques:
\vspace{3em}
\item Maximal paths:
\vspace{3em}
\item Maximal matchings:
\vspace{3em}
\end{enumerate}
\end{minipage}
\begin{minipage}[t]{0.4\textwidth}
\begin{center}
\begin{graph}
\foreach \i in {0,1,2}
{
  \node[vertex] (v\i) at (120*\i:1) {};
}
\node[vertex,label={$d$}] (w) at (0:2.5) {};
\node[label={$a$}] at (v1) {};
\node[label=below:{$b$}] at (v2) {};
\node[label={$c$}] at (v0) {};
\draw (w)--(v0)--(v1)--(v2)--(v0);
\end{graph}
\end{center}
\end{minipage}

%7.1.1
\question[16] Determine $\chi'(G)$ and $\chi'(H)$.  \textit{(A theorem gives the answer for $H$ quickly, but $G$ requires some argument.)}
\begin{center}
\begin{graph}
\foreach \i in {0,1,2}
{
  \node[vertex] (v\i) at (90+120*\i:2) {};
  \node[vertex] (w\i) at (30+120*\i:1) {};
}
\draw (w0)--(v0)--(w1)--(v1)--(w2)--(v2)--(w0);
\draw (w0)--(w1)--(w2)--(w0);
\node at (0,-1.75) {$G$};
\end{graph}
\qquad\qquad
\begin{graph}
\foreach \i in {0,1,2}
{
  \node[vertex] (u\i) at (2*\i,0) {};
  \node[vertex] (v\i) at (2*\i,2.5) {};
}
\foreach \i in {0,1,2}
{
  \foreach \j in {0,1,2}
  {
    \draw (u\i) to[bend left=10] (v\j);
    \draw (u\i) to[bend right=10] (v\j);
  }
}
\node at (2,-0.75) {$H$};
\end{graph}
\end{center}

\pagebreak

\question[18] For each of the following, determine whether such a graph exists.  If it does, give an example; if not, give a brief reason why not.
\begin{parts}
\part a 3-connected 4-regular planar graph
\vspace{10em}
\part a 4-connected 3-regular planar graph
\vspace{10em}
\part a 6-connected planar graph
\vspace{10em}
\end{parts}
%\question Use the Four Color Theorem to prove that every planar graph decomposes into two bipartite graphs.  \textit{(Hint: start with a coloring, and consider for each edge the sum of its endpoints' colors.)}
\question[10] Decompose the following planar graph into two bipartite graphs.  \textit{(Hint: this can always be done, using the Four Color Theorem and considering, for each edge, the sum its endpoints' colors.)}
\begin{center}
\begin{graph}[scale=2]
%\foreach \i in {0,1,...,4}
%{
%  \node[vertex] (v\i) at (90+72*\i:3) {};
%}
%\foreach \j in {1,2,3}
%{
%  \node[vertex] (u\j) at (270+120*\j:1) {};
%}
\node[vertex] (v0) at (1,1.5) {};
\node[vertex] (v1) at (0,2) {};
\node[vertex] (v2) at (0,0) {};
\node[vertex] (v3) at (2,0) {};
\node[vertex] (v4) at (2,2) {};
\node[vertex] (u1) at (1.5,1) {};
\node[vertex] (u2) at (0.5,1) {};
\node[vertex] (u3) at (1,0.5) {};
\draw (v0)--(v1)--(v2)--(v3)--(v4)--(v0);   \draw (u1)--(u2)--(u3)--(u1);
\draw (v4)--(u1)--(v0)--(u2)--(v1);
\draw (u2)--(v2)--(u3)--(v3)--(u1);
\draw (v1)--(v4);
\end{graph}
%OCTAHEDRON:   (3-colorable, meh)
%\begin{graph}
%\foreach \i in {0,1,2}
%{
%  \node[vertex] (w\i) at (30+120*\i:0.5) {};
%  \node[vertex] (v\i) at (90+120*\i:2) {};
%}
%\draw (w0)--(w1)--(w2)--(w0);  \draw (v0)--(v1)--(v2)--(v0);
%\draw (w0)--(v0)--(w1)--(v1)--(w2)--(v2)--(w0);
%\end{graph}
\end{center}
\vfill
\pagebreak

\setcounter{savedqn}{\thequestion}
\end{questions}
\section*{Algorithms}
Give brief justifications for your answers (but not necessarily full proofs; a computation with each step clearly written may suffice).
\begin{questions}
\setcounter{question}{\thesavedqn}


\question Consider the degree sequence $6,6,6,4,3,3,3,3$.
\begin{parts}
\part[10] Prove that every simple graph with this degree sequence is 4-colorable.
\vfill
\part[10] Find one simple graph with this degree sequence and a 4-coloring of it.
\vfill
\part (8 \textbf{bonus}) Prove that every simple graph with this degree sequence has chromatic number at least~3.
\vfill
\end{parts}
%\leq 4 by greedy; cannot be bipartite because only one degree not divisible by 3
%oh, or: it would have 17>16=4*4 edges
%is graphic?  5533222, 422211, 11110, yep

\pagebreak

\question[10] Let $G$ be a weighted graph in which every edge has positive weight, and let $T^*$ be a minimum spanning tree.  An edge $e$ with weight zero is added to $G$; describe how to modify $T^*$ to obtain a minimum spanning tree of $G+e$.%  (Briefly justify.)

\pagebreak

\question[12] Apply the Ford-Fulkerson algorithm to augment the feasible flow below to a maximum feasible flow.  (Explain how you know the result is maximum.)  \textit{(The network is redrawn for your convenience.)}

\begin{graph}[xscale=3,yscale=2]
\tikzset{->-/.style={decoration={
  markings,
  mark=at position #1 with {\arrow{Latex[scale=1.5]}}},postaction={decorate}}}
\node[vertex,label=left:{$s$}] (s) at (0,0) {};
\node[vertex] (a1) at (1,1) {};  \node[vertex] (a2) at (1,0) {};  \node[vertex] (a3) at (1,-1) {};
\node[vertex] (b1) at (2,1.5) {}; \node[vertex] (b2) at (2,0.5) {}; \node[vertex] (b3) at (2,-.5) {}; \node[vertex] (b4) at (2,-1.5) {};
\node[vertex] (c1) at (3,1) {}; \node[vertex] (c2) at (3,0) {}; \node[vertex] (c3) at (3,-1) {};
\node[vertex,label=right:{$t$}] (t) at (4,0) {};
\foreach \t/\h/\f/\c/\p in {%tail / head / flow / capacity
  s/a1/4/4/above left,
  s/a2/2/6/above,
  s/a3/3/5/below left,
  a1/b1/3/3/above,
  a1/a2/1/1/left,
  a2/b1/1/2/left,
  a2/b2/0/1/above,
  a2/b3/2/5/below,
  a2/a3/0/2/above left,
  a3/b3/0/4/below right,
  a3/b4/3/5/below left,
  b1/c1/4/4/above,
  b2/c1/1/1/above,
  b2/c2/0/6/below,
  b3/b2/1/4/left,
  b3/c2/0/2/below right,
  b3/c3/1/1/below left,
  b4/c3/3/3/below right,
  c1/c2/1/2/left,
  c1/t/4/6/above right,
  c2/t/1/1/above,
  c3/t/4/7/below right}
{  \draw[->-=0.6] (\t) to node[\p]{$\f/\c$} (\h);  }
\end{graph}

\vfill
\begin{graph}[xscale=3,yscale=2]
\tikzset{->-/.style={decoration={
  markings,
  mark=at position #1 with {\arrow{Latex[scale=1.5]}}},postaction={decorate}}}
\node[vertex,label=left:{$s$}] (s) at (0,0) {};
\node[vertex] (a1) at (1,1) {};  \node[vertex] (a2) at (1,0) {};  \node[vertex] (a3) at (1,-1) {};
\node[vertex] (b1) at (2,1.5) {}; \node[vertex] (b2) at (2,0.5) {}; \node[vertex] (b3) at (2,-.5) {}; \node[vertex] (b4) at (2,-1.5) {};
\node[vertex] (c1) at (3,1) {}; \node[vertex] (c2) at (3,0) {}; \node[vertex] (c3) at (3,-1) {};
\node[vertex,label=right:{$t$}] (t) at (4,0) {};
\foreach \t/\h/\f/\c/\p in {%tail / head / flow / capacity
  s/a1/4/4/above left,
  s/a2/2/6/above,
  s/a3/3/5/below left,
  a1/b1/3/3/above,
  a1/a2/1/1/left,
  a2/b1/1/2/left,
  a2/b2/0/1/above,
  a2/b3/2/5/below,
  a2/a3/0/2/above left,
  a3/b3/0/4/below right,
  a3/b4/3/5/below left,
  b1/c1/4/4/above,
  b2/c1/1/1/above,
  b2/c2/0/6/below,
  b3/b2/1/4/left,
  b3/c2/0/2/below right,
  b3/c3/1/1/below left,
  b4/c3/3/3/below right,
  c1/c2/1/2/left,
  c1/t/4/6/above right,
  c2/t/1/1/above,
  c3/t/4/7/below right}
{  \draw[->-=0.6] (\t) to node[\p]{$\,\,/\c$} (\h);  }
\end{graph}

\vfill
\begin{graph}[xscale=3,yscale=2]
\tikzset{->-/.style={decoration={
  markings,
  mark=at position #1 with {\arrow{Latex[scale=1.5]}}},postaction={decorate}}}
\node[vertex,label=left:{$s$}] (s) at (0,0) {};
\node[vertex] (a1) at (1,1) {};  \node[vertex] (a2) at (1,0) {};  \node[vertex] (a3) at (1,-1) {};
\node[vertex] (b1) at (2,1.5) {}; \node[vertex] (b2) at (2,0.5) {}; \node[vertex] (b3) at (2,-.5) {}; \node[vertex] (b4) at (2,-1.5) {};
\node[vertex] (c1) at (3,1) {}; \node[vertex] (c2) at (3,0) {}; \node[vertex] (c3) at (3,-1) {};
\node[vertex,label=right:{$t$}] (t) at (4,0) {};
\foreach \t/\h/\f/\c/\p in {%tail / head / flow / capacity
  s/a1/4/4/above left,
  s/a2/2/6/above,
  s/a3/3/5/below left,
  a1/b1/3/3/above,
  a1/a2/1/1/left,
  a2/b1/1/2/left,
  a2/b2/0/1/above,
  a2/b3/2/5/below,
  a2/a3/0/2/above left,
  a3/b3/0/4/below right,
  a3/b4/3/5/below left,
  b1/c1/4/4/above,
  b2/c1/1/1/above,
  b2/c2/0/6/below,
  b3/b2/1/4/left,
  b3/c2/0/2/below right,
  b3/c3/1/1/below left,
  b4/c3/3/3/below right,
  c1/c2/1/2/left,
  c1/t/4/6/above right,
  c2/t/1/1/above,
  c3/t/4/7/below right}
{  \draw[->-=0.6] (\t) to node[\p]{$\,\,/\c$} (\h);  }
\end{graph}


\pagebreak

\question[12] Find the maximum number of edges in a bipartite planar subgraph of the Petersen graph. \\  \textit{(Hint: use the idea behind the generic planar edge bound; what face lengths can such a subgraph have?  For the construction, use that information about face lengths to choose a good drawing of the Petersen graph to modify.  The three most common drawings are given at the bottom of the page.)}

\vfill

\begin{center}
\begin{graph}
\tikzstyle{every path}=[dotted]
\foreach \i in {1,2,3,4,5}
{
  \node[vertex] (u\i) at (90+72*\i:1) {};
  \node[vertex] (v\i) at (90+72*\i:2) {};
  \draw (u\i)--(v\i);
}
\draw (u1) to (u3) to (u5) to (u2) to (u4) to (u1);
\draw (v1)--(v2)--(v3)--(v4)--(v5)--(v1);
\end{graph}
\qquad\qquad
\begin{graph}
\tikzstyle{every path}=[dotted]
\node[vertex] (w) at (0,0) {};
\foreach \i in {1,2,3,4,5,6}
{
  \node[vertex] (v\i) at (-60+60*\i:2) {};
  \node[vertex] (u\i) at (30-120*\i:1) {};
  \draw (v\i)--(u\i)--(w);
}
\draw (v1)--(v2)--(v3)--(v4)--(v5)--(v6)--(v1);
\end{graph}
\qquad\qquad
\begin{graph}
\tikzstyle{every path}=[dotted]
\node[vertex] (w) at (0,0) {};
\foreach \i in {1,2,...,9}
{
  \node[vertex] (v\i) at (50+40*\i:2) {};
}
\draw (v1)--(v2)--(v3)--(v4)--(v5)--(v6)--(v7)--(v8)--(v9)--(v1);
\draw (w)--(v1); \draw (w)--(v4); \draw (w)--(v7);
\draw (v2)--(v6); \draw (v3)--(v8); \draw (v5)--(v9);
\end{graph}
\end{center}

\pagebreak


\setcounter{savedqn}{\thequestion}
\end{questions}
%\pagebreak
\section*{Proofs}
\begin{questions}
\setcounter{question}{\thesavedqn}


%\part A graph $G$ is bipartite if and only if $G\cp K_2$ is bipartite.
%\part Find (with proof) the largest number of edges in a $C_4$-free subgraph of $Q_3$.

\question[20] Let $G$ be $k$-connected, $x\in V(G)$, and $U\subseteq V(G)\setminus\{x\}$ with $|U|=k$.  Prove that there are $k$ paths from $x$ to $U$ that share only the vertex $x$.  \textit{(Hint: Use the Expansion Lemma and Menger's Theorem.)}

\pagebreak

\question[20] Prove that every outerplanar graph is 3-colorable.  \textit{(Hint: there is an easy proof using the Four Color Theorem (a bit of overkill), and another proof similar to that of the Six Color Theorem.)}
%from 4 color theorem, by adding a dominating vertex  OR  degeneracy


\pagebreak
\question[20] Prove that for $k\geq2$, the hypercube $Q_k$ is Hamiltonian.
%\question[20] If $G$ and $H$ are both Hamiltonian, then $G\cp H$ is also Hamiltonian.  \textit{(Hint: name/draw the vertices of $G$ and $H$ in order along their Hamiltonian cycles.  Consider cases for the parity of the number of vertices in $H$.)}

%\question %3.1.19
%....system of distinct representatives....


%(The next are probably better as T/F, with just short justification)
%\question %5.1.12
%Prove or disprove: every optimal coloring of $G$ has a color class of size $\alpha(G)$.
%Prove or disprove: every $k$-chromatic graph $G$ has a proper $k$-coloring with a color class of size $\alpha(G)$.
%\question %5.1.15
%Prove or disprove: if $a$ is the average degree of $G$, then $\chi(G)\leq 1+a$.

\end{questions}




%\ifprintanswers{}\else{
%\newpage
%\textbf{Scratch Paper - Do Not Remove}
\newpage
\textbf{Scratch Paper} - you may remove this if you find it convenient
\newpage
\textbf{Scratch Paper} - you may remove this if you find it convenient
%\newpage
%\textbf{Scratch Paper} - you may remove this if you find it convenient
%\newpage
%\textbf{Scratch Paper} - you may remove this if you find it convenient
%\newpage\hspace{-1em}\includegraphics{Ch1Tables.pdf}
%} \fi





\end{document}

