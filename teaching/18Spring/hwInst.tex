\documentclass[11pt]{amsart}
%\usepackage{hyperref}
\usepackage[margin=1in]{geometry}

\usepackage{enumitem}
\setlist[itemize]{noitemsep, topsep=0pt}

%\hyphenation{WebAssign}%don't linebreak and hyphenate WebAssign

\begin{document}
\title{Homework Instructions \\ Math 454 }
\maketitle
\vspace{-2.2em}\begin{center}\small Ben Reiniger\footnote{adapted from Hemanshu Kaul}
\end{center}
\thispagestyle{empty}
\vspace{1em}

Homework serves as an opportunity for students to practice 
communicating written mathematics with clarity of thought and language.  In any course like
this, learning good communication skills in mathematics is very important. As significant is the
opportunity that a homework provides you to test your understanding of the material covered in
class that week. Mathematics cannot be learned by listening or just reading a book -- you have to
\textbf{do} it. Considering the varying pace of learning of students in class and the lack of class time to explore every detail of every concept/theorem, working through problems in the HW is an easy way for you to make sure that you are keeping up with the class. This is why homework is given a lot of importance in this course -- dedicate enough time to it every week.  (A general rule of thumb is to give yourself 2 hours of study time per course hour; here that's about 6 hours a week, most of which should be spent doing homework.  Depending on how quickly you grasp the material, you may well need more time.)

Some of the homework problems will be straightforward applications of the definitions or theorems studied in class, however every homework will also contain some challenging problems. Don't be disheartened if some problems take a while to solve. Such problems help develop your mathematical creativity. Discuss such problems with your classmates, and/or ask me for help, but only after you have given them sufficient thought. Please remember that
\textbf{homework is NOT meant to be an examination; it is meant to assist in your learning and development. If you need help with it, don't hesitate to ask.}

To improve your mathematical writing quickly, start by writing draft solutions to homework
early. A day or two later after you have had time to forget what you wrote, read it. If it doesn't make sense or convince you, rewrite it. Writing a solution requires saying what you mean and meaning what you say.  Be intellectually honest.  Intellectual dishonesty includes: (1) stating a ``reason'' without understanding its relevance;  (2) claiming a conclusion when you know you haven't proved it; and (3) giving an example and claiming you have proved the statement for all instances. Include enough detail in your solutions so that your explanation is convincing to someone who hasn't thought about the problem before. The proofs/arguments should be presented so that your classmates could read them and follow the logic (step-by-step).

Any incident of plagiarism/cheating (from a person or from any online resource)
will be strictly dealt with.
Solutions may not be sought from solution manuals or any other source of full written solutions (whether printed, web-based, or otherwise). You are allowed to discuss homework problems with your classmates, the instructor, and Applied Math department tutors. However, the \textbf{solutions should be written by you alone}.  Solutions for homework and exams must be written clearly, legibly, and concisely, and will be graded for both mathematical correctness and presentation. Points will be deducted for sloppiness, rambling, incoherent or insufficient explanation, or lack of supporting rationale.


You are encouraged to ask questions during class, or in office hours, or through email.
If you are having trouble solving a homework problem, I will be glad to put you in the right
direction. The same goes for any concept/proof you have difficulty understanding. Don't hesitate
to ask for help! I cannot help you if you don't take the initiative. 
You are also strongly encouraged to attend the weekly problem-solving sessions, where you can ask your peers and me questions about the difficulties you have faced while solving that week's assignment.



\end{document}