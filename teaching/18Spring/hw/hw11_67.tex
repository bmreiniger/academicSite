\documentclass[11pt]{amsart}%10pt for solutions
\usepackage[margin=1in]{geometry}
  
\usepackage{amsmath,amsthm,amssymb}
\usepackage{enumerate}
\usepackage{tikz}
%\usepackage{nth}

\newcommand{\cp}{\mathbin\Box}
\newcommand{\ceil}[1]{\left\lceil#1\right\rceil}
\newcommand{\floor}[1]{\left\lfloor#1\right\rfloor}
\DeclareMathOperator{\rad}{rad}
\DeclareMathOperator{\diam}{diam}


\newenvironment{graph}[1][scale=1]{
\begin{tikzpicture}[#1]
\tikzstyle{vertex}=[circle, draw, fill, inner sep=0pt, minimum size=4pt]%
\tikzstyle{bigvtx}=[circle, draw, fill, inner sep=0pt, minimum size=6pt]%
\tikzstyle{every path}=[line width=0.5pt]%
}{\end{tikzpicture}}


\begin{document}
\title{Math 454\\ Homework 11 %Solutions 
\qquad Due April 26*%<---removeforsolutions
}
\author{}
\date{}
\maketitle
\thispagestyle{empty}

%remove for solutions:

\noindent Name:~\hrulefill~~\\

\begin{itemize}
\item Refer to the syllabus regarding allowed collaboration on this homework assignment.
\item Refer to other homework instructions and suggestions posted in Blackboard.
\item All answers must be fully justified.
\item Your homework should be neatly written on additional paper; you may attach this cover page if you would like to keep the questions attached to the answers.
\end{itemize}

\bigskip
Turn in four of the following problems to be graded.
\bigskip
%-----/remove for solutions

\begin{enumerate}
\item (6.1.14) Prove or disprove: for every $n\in\mathbb{N}$, there is a simple connected 4-regular planar graph with at least $n$ vertices.

\item (6.2.5) Determine the maximum number of edges in a planar subgraph of the Petersen graph.

\item (6.2.7) \textit{(A graph is \emph{outerplanar} if it has a planar drawing in which all vertices lie on the unbounded face.)}  Use Kuratowski's Theorem to prove that $G$ is outerplanar if and only if it has no subgraph that is a subdivision of $K_4$ or $K_{2,3}$.
%\item (6.2.8? no, needs fan lemma)
%\item 6.2.12? no, weird/easy

%\item (6.3.5) EXAM INSTEAD: Use the Four Color Theorem to prove that every planar graph decomposes into two bipartite subgraphs.
\item (6.3.14, ``only if'') Prove that every 3-colorable plane triangulation is Eulerian.  \textit{(Hint: consider the dual)}

%\item (7.1.15)

\item (7.1.26) Let $G$ be a regular graph with a cut-vertex.  Prove that $\chi'(G)>\Delta(G)$.

\item (7.2.8) On a chessboard, a knight can move from one square to another that differs by 1 in one coordinate and by 2 in the other coordinate.  A \emph{knight's tour} is a traversal of the board by a knight in which each square is visited exactly once, except that the knight returns to its starting square.  Prove that no $4\times n$ chessboard has a knight's tour.  \textit{(Hint: find a set of vertices in the corresponding graph that violates the necessary condition for a Hamiltonian cycle.  There is also an alternative proof using a very early result of ours in a clever way.)}

%\item (7.2.9?

\item (7.3.7) Let $G$ be a plane triangulation.
\begin{enumerate}
\item Prove that the dual $G^*$ has a 2-factor.
\item Use part (a) to prove that the vertices of $G$ can be 2-colored (not properly) so that every face has vertices of both colors.  \textit{(Hint: use the idea in the proof of Theorem~7.3.2.)}
\end{enumerate}

\end{enumerate}


\end{document}

