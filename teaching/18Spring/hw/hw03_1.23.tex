\documentclass[11pt]{amsart}
    \topmargin -.5 in
    \textheight 9.0in
    \textwidth 6.25 in
    \oddsidemargin 0 in
    \evensidemargin 0 in
   
\usepackage{amsmath}
\usepackage{enumerate}
\usepackage{tikz}
%\usepackage{nth}

\newcommand{\ceil}[1]{\left\lceil#1\right\rceil}
\newcommand{\floor}[1]{\left\lfloor#1\right\rfloor}


\newenvironment{graph}[1][scale=1]{
\begin{tikzpicture}[#1]
\tikzstyle{vertex}=[circle, draw, fill, inner sep=0pt, minimum size=4pt]%
\tikzstyle{bigvtx}=[circle, draw, fill, inner sep=0pt, minimum size=6pt]%
\tikzstyle{every path}=[line width=0.5pt]%
}{\end{tikzpicture}}


\begin{document}
\title{Math 454\\ Homework 3 \qquad Due February 8}
\author{}
\date{}
\maketitle
\thispagestyle{empty}

\noindent Name:~\hrulefill~~\\

\begin{itemize}
\item Refer to the syllabus regarding allowed collaboration on this homework assignment.
\item Refer to other homework instructions and suggestions posted in Blackboard.
\item All answers must be fully justified.
\item Your homework should be neatly written on additional paper; you may attach this cover page if you would like to keep the questions attached to the answers.
\end{itemize}

\bigskip

Turn in four of the following problems to be graded.  You \emph{must} choose at least one of the hypercube problems.

\bigskip

Solve at least one of the following two hypercube problems.
\begin{enumerate}[Q1]
\item (1.3.26)   Count the 6-cycles in $Q_3$.  Prove that every 6-cycle in $Q_k$ ($k\geq3$) lies in exactly one 3-dimensional subcube.  Use this to count the 6-cycles in $Q_k$ for $k\geq3$.
\item (1.3.27)   Given $k\in\mathbb{N}$, let $G$ be the subgraph of $Q_{2k+1}$ induced by vertices in which the number of ones is $k$ or $k+1$.  Prove that $G$ is regular, and compute $n(G)$, $e(G)$, and the girth of $G$.
\end{enumerate}

\bigskip

%MOVED TO IN-CLASS: 
%Solve at least one of the following two challenging problems.
%\begin{enumerate}[C1]
%\item (1.3.16)  For $k\geq2$ and $g\geq2$, prove that there exists a $k$-regular graph with girth $g$.  (Use induction on $g$.  Within that proof, use induction on $k$.  In the inductive step, make use of a $(k-1)$-regular graph $H$ with girth $g$ and an $n(H)$-regular graph $G$ with girth $\ceil{g/2}$.)
%\item (1.3.19)  Let $G$ be a claw-free simple graph.  Prove that if $\Delta(G)\geq5$, then $G$ has a 4-cycle.  Prove that this is sharp, by constructing a 4-regular claw-free graph having no 4-cycle.  (For the construction:  use your proof for $\Delta(G)\geq5$ to describe what such a graph must look like locally, and find a correspondence with a certain class of 3-regular graphs.)
%\end{enumerate}
%\bigskip

\begin{enumerate}[P1]
\item 
\begin{enumerate}
%\item 1.1.10.  Prove that if $G$ is disconnected, then $\overline{G}$ is connected.
\item (1.3.3)   Let $u$ and $v$ be adjacent vertices in a simple graph $G$.  Prove that $uv$ belongs to at least $d(u)+d(v)-n(G)$ triangles of $G$.
\item (1.3.41) Prove that if $G$ is an $n$-vertex simple graph with $\Delta(G)=\ceil{n/2}$ and $\delta(G)=\floor{n/2}-1$, then $G$ is connected.
\end{enumerate}

\item (1.3.9)  In a league with two divisions of 13 teams each, determine whether it is possible to schedule a season with each team playing nine games against teams within its division and four games against teams in the other division.  (Clearly state your graph theoretic model for this problem.)
\item (1.3.17)  Let $G$ be a graph with at least two vertices.  Prove or disprove:
\begin{enumerate}
\item Deleting a vertex of degree $\Delta(G)$ cannot increase the average degree.
\item Deleting a vertex of degree $\delta(G)$ cannot decrease the average degree.
\end{enumerate}
\item (1.3.32)  Prove that the number of simple even graphs with vertex set $[n]$ is $\displaystyle 2^{\binom{n-1}{2}}$.  (Hint: find a bijection with the set of simple graphs with vertex set $[n-1]$.)

\end{enumerate}


\vfill
\begin{quotation}
\footnotesize
\end{quotation}

\end{document}

