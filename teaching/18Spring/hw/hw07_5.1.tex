\documentclass[11pt]{amsart}%10pt for solutions
\usepackage[margin=1in]{geometry}
  
\usepackage{amsmath,amsthm,amssymb}
\usepackage{enumerate}
\usepackage{tikz}
%\usepackage{nth}

\newcommand{\cp}{\mathbin\Box}
\newcommand{\ceil}[1]{\left\lceil#1\right\rceil}
\newcommand{\floor}[1]{\left\lfloor#1\right\rfloor}
\DeclareMathOperator{\rad}{rad}
\DeclareMathOperator{\diam}{diam}


\newenvironment{graph}[1][scale=1]{
\begin{tikzpicture}[#1]
\tikzstyle{vertex}=[circle, draw, fill, inner sep=0pt, minimum size=4pt]%
\tikzstyle{bigvtx}=[circle, draw, fill, inner sep=0pt, minimum size=6pt]%
\tikzstyle{every path}=[line width=0.5pt]%
}{\end{tikzpicture}}


\begin{document}
\title{Math 454\\ Homework 7 %Solutions 
\qquad Due March 22%<---removeforsolutions
}
\author{}
\date{}
\maketitle
\thispagestyle{empty}

%remove for solutions:
\noindent Name:~\hrulefill~~\\

\begin{itemize}
\item Refer to the syllabus regarding allowed collaboration on this homework assignment.
\item Refer to other homework instructions and suggestions posted in Blackboard.
\item All answers must be fully justified.
\item Your homework should be neatly written on additional paper; you may attach this cover page if you would like to keep the questions attached to the answers.
\end{itemize}

\bigskip
Turn in four of the following problems to be graded.
\bigskip
%-----/remove for solutions

\begin{enumerate}
\item (5.1.19)  \emph{spot the error in a false proof; see textbook for statement}
\item (5.1.20)  Let $G$ be a graph whose odd cycles are pairwise intersecting, i.e.~every two odd cycles have a common vertex.  Prove that $\chi(G)\leq5$.
\item (5.1.24)  Let $G$ be any 20-regular 360-vertex graph with the following properties: the vertices are evenly spaced around a circle; vertices separated by 1 or 2 degrees (along the circle) are nonadjacent, while vertices separated by 3, 4, 5, or 6 degrees are adjacent.  (Vertices separated by more than 6 degrees may or may not be adjacent, subject to the property that $G$ is 20-regular.)  Prove that $\chi(G)\leq19$.  \textit{(Remark: Brooks's Theorem gives $\chi(G)\leq20$.  Hint: color the vertices successively along the circle.)}
\item (5.1.34)  For all $k$, construct a tree $T_k$ with maximum degree $k$ and an ordering $\sigma$ of $V(T_k)$ such that the greedy coloring relative to $\sigma$ uses $k+1$ colors.  \textit{(Hint: use induction on $k$, and construct the tree and the ordering together.)}
\item (5.1.31)  Prove that $G$ is $m$-colorable if and only if $G\cp K_m$ has an independent set of size~$|V(G)|$.
\item (8.4.21) Prove that $K_{k,m}$ is $k$-choosable if and only if $m<k^k$.
\end{enumerate}


\end{document}

