\documentclass[11pt]{amsart}%10pt for solutions
\usepackage[margin=1in]{geometry}
  
\usepackage{amsmath,amsthm,amssymb}
\usepackage{enumerate}
\usepackage{tikz}
%\usepackage{nth}

\newcommand{\cp}{\mathbin\Box}
\newcommand{\ceil}[1]{\left\lceil#1\right\rceil}
\newcommand{\floor}[1]{\left\lfloor#1\right\rfloor}
\DeclareMathOperator{\rad}{rad}
\DeclareMathOperator{\diam}{diam}


\newenvironment{graph}[1][scale=1]{
\begin{tikzpicture}[#1]
\tikzstyle{vertex}=[circle, draw, fill, inner sep=0pt, minimum size=4pt]%
\tikzstyle{bigvtx}=[circle, draw, fill, inner sep=0pt, minimum size=6pt]%
\tikzstyle{every path}=[line width=0.5pt]%
}{\end{tikzpicture}}


\begin{document}
\title{Math 454\\ Homework 8 %Solutions 
\qquad Due March 29%<---removeforsolutions
}
\author{}
\date{}
\maketitle
\thispagestyle{empty}

%remove for solutions:
\noindent Name:~\hrulefill~~\\

\begin{itemize}
\item Refer to the syllabus regarding allowed collaboration on this homework assignment.
\item Refer to other homework instructions and suggestions posted in Blackboard.
\item All answers must be fully justified.
\item Your homework should be neatly written on additional paper; you may attach this cover page if you would like to keep the questions attached to the answers.
\end{itemize}

\bigskip
Turn in four of the following problems to be graded.
\bigskip
%-----/remove for solutions

\begin{enumerate}
\item
\begin{enumerate}
\item (5.2.3, part) Prove that if $G$ and $H$ are both color-critical, then so is $G\vee H$.
\item (5.2.9) Prove that if $G$ is $k$-critical, then $M(G)$ (the Mycielskian of $G$) is $(k+1)$-critical.
\end{enumerate}

\item (5.2.32, part) \textit{The Haj\'os construction.} Let $G$ and $H$ be $k$-critical graphs sharing only vertex~$v$, with $vu\in E(G)$ and $vw\in E(H)$.  Prove that the graph $F=(G-vu)\cup(H-vw)\cup uw$ is also $k$-critical.  (The construction is illustrated below.)
\begin{center}
\begin{graph}
\draw (-1,0) circle (1); \node at (-1,-1.3) {$G$};
\draw (1,0) circle (1); \node at (1,-1.3) {$H$};
\node[vertex] (v) at (0,0) {}; \node at (0.2,-0.1) {$v$};
\node[vertex] (u) at (-0.5,0.5) {}; \node at (-.7,0.4) {$u$};
\node[vertex] (w) at (0.5,0.5) {}; \node at (0.75,0.4) {$w$};
\draw (u)--(v)--(w);
\begin{scope}[xshift=3cm]
\draw[->] (-.5,0)--(.5,0);
\end{scope}
\begin{scope}[xshift=6cm]
\draw (-1,0) circle (1); 
\draw (1,0) circle (1); 
\node[vertex] (v) at (0,0) {}; \node at (0.2,-0.1) {$v$};
\node[vertex] (u) at (-0.5,0.5) {}; \node at (-.7,0.4) {$u$};
\node[vertex] (w) at (0.5,0.5) {}; \node at (0.75,0.4) {$w$};
\draw[dashed] (u)--(v)--(w);
\draw (u)--(w);
\node at (0,-1.3) {$F$};
\end{scope}
\end{graph}
\end{center}

\item (5.2.15) Prove that every triangle-free $n$-vertex graph has chromatic number at most $2\sqrt{n}$.  \textit{(Hint: iteratively color large neighborhoods while they exist, then apply Brooks' Theorem.)  (See also Remark~5.2.4.)}

\item (5.2.29) Let $G$ be a claw-free graph (no induced $K_{1,3}$).
\begin{enumerate}
\item Prove that the subgraph induced by the union of any two color classes in a proper coloring of $G$ consists of paths and even cycles.
\item Prove that if $G$ has a proper coloring using exactly $k$ colors, then $G$ has a proper $k$-coloring in which the color classes differ in size by at most one.
\end{enumerate}

\item \textit{``Continuity'' of (list-)chromatic number.}  Let $G$ be a graph with a vertex $v$.
\begin{enumerate}
\item Prove that $\chi(G)\leq \chi(G-v)+1$.
\item Prove that $\chi_{\ell}(G)\leq\chi_{\ell}(G-v)+1$.
\end{enumerate}
%\item (5.1.41, 8.4.22) Prove that $\chi_{\ell}(G)+\chi_{\ell}(\overline{G})\leq n(G)+1$.

\item Prove that every even cycle is 2-choosable.  \textit{(Hint: consider separately the case that every vertex receives the same list.  If that's not the case, then there are two adjacent vertices with different lists (why?); now color around the cycle starting with one of these.)}
\end{enumerate}


\end{document}

