\documentclass[11pt]{amsart}%10pt for solutions
\usepackage[margin=1in]{geometry}
  
\usepackage{amsmath,amsthm,amssymb}
\usepackage{enumerate}
\usepackage{tikz}
%\usepackage{nth}

\newcommand{\ceil}[1]{\left\lceil#1\right\rceil}
\newcommand{\floor}[1]{\left\lfloor#1\right\rfloor}
\DeclareMathOperator{\rad}{rad}
\DeclareMathOperator{\diam}{diam}


\newenvironment{graph}[1][scale=1]{
\begin{tikzpicture}[#1]
\tikzstyle{vertex}=[circle, draw, fill, inner sep=0pt, minimum size=4pt]%
\tikzstyle{bigvtx}=[circle, draw, fill, inner sep=0pt, minimum size=6pt]%
\tikzstyle{every path}=[line width=0.5pt]%
}{\end{tikzpicture}}


\begin{document}
\title{Math 454\\ Homework 6 %Solutions 
\qquad Due March 1%<---removeforsolutions
}
\author{}
\date{}
\maketitle
\thispagestyle{empty}

%remove for solutions:
\noindent Name:~\hrulefill~~\\

\begin{itemize}
\item Refer to the syllabus regarding allowed collaboration on this homework assignment.
\item Refer to other homework instructions and suggestions posted in Blackboard.
\item All answers must be fully justified.
\item Your homework should be neatly written on additional paper; you may attach this cover page if you would like to keep the questions attached to the answers.
\end{itemize}

\bigskip
Turn in four of the following problems to be graded.
\bigskip
%-----/remove for solutions

\begin{enumerate}
%\item (2.2.31 ish) graceful caterpillars
\item (2.2.23) Assume for this problem that the Graceful Tree Conjecture is true.  Let $T$ be a tree with $m$ edges.  Show that $K_{2m}$ decomposes into copies of~$T$.  \textit{(Hint: modify the construction that decomposes $K_{2m-1}$ into copies of a tree with $m-1$ edges.)}
\item (2.2.10, first part) Find $\tau(K_{2,n})$.
\item (2.2.12) From a graph $G$, let $G^{(k)}$ be obtained by replacing each edge by $k$ copies of that edge; and let $G^{1/k}$ be obtained by replacing each edge $uv$ by a $u,v$-path of length $k$ through $k-1$ new vertices.  Determine $\tau(G^{(k)})$ and $\tau(G^{1/k})$ in terms of $\tau(G)$, $k$, $n(G)$, and $e(G)$.  \textit{(Below are examples of these constructions.)}
\begin{center}
\begin{graph}
\foreach \i in {1,2,3,4} {\node[vertex] (v\i) at (135-90*\i:1) {}; }
\foreach \i/\j in {1/2,2/3,3/4,4/1,1/3}
{
  \draw (v\i)--(v\j);
}
\draw (v3) to[bend left] (v4);
\node at (0,-1.25) {$G$};
\end{graph}
\qquad
\begin{graph}
\foreach \i in {1,2,3,4} {\node[vertex] (v\i) at (135-90*\i:1) {}; }
\foreach \i/\j in {1/2,2/3,3/4,4/1,1/3}
{
  \draw (v\i) to[bend left](v\j);
  \draw (v\i) to (v\j);
}
\draw (v3) to[bend right] (v4);
\draw (v3) to[out=150, in=210] (v4);
\node at (0,-1.25) {$G^{(2)}$};
\end{graph}
\qquad
\begin{graph}
\foreach \i in {1,2,3,4} {\node[vertex] (v\i) at (135-90*\i:1) {}; }
\foreach \i/\j in {1/2,2/3,3/4,4/1,1/3}
{
  \draw (v\i) -- (v\j) node[midway,vertex] {};
}
\draw (v3) to [bend left] node[midway,vertex] {} (v4);
\node at (0,-1.25) {$G^{1/2}$};
\end{graph}
\end{center}
\item (2.3.10) \textbf{Prim's Algorithm} grows a spanning tree from a given vertex of a connected weighted graph $G$, iteratively adding the cheapest edge from a vertex already reached to a vertex not yet reached, finishing when all the vertices of $G$ have been reached.  Prove that Prim's Algorithm produces a minimum-weight spanning tree of $G$.
\item (2.3.12) \textit{Minimum spanning path.}  In a weighted complete graph, iteratively select the edge of least weight such that the edges selected so far form a disjoint union of paths.  After $n-1$ steps, the result is a spanning path.  This greedy algorithm \emph{does not} always produce a minimum-weight spanning path; provide an infinite family of weighted complete graphs for which the algorithm produces a suboptimal spanning path.
\item (2.3.16) Four people must cross a canyon at night on a fragile bridge.  At most two people can be on the bridge at once.  Crossing requires carrying a flashlight, and there is only one flashlight (which can only cross by being carried).  Alone, the four people cross in 10, 5, 2, 1 minutes, respectively.  When two cross together, they move at the speed of the slower person.  In 18 minutes, a flash flood coming down the canyon will wash away the bridge.  Can the four people get across in time?  Describe how the answer can be found using graph theory.

\end{enumerate}


\end{document}

